% Created 2012-07-02 Mon 07:11
\documentclass[12pt]{article}
\usepackage[utf8]{inputenc}
\usepackage[T1]{fontenc}
\usepackage{fixltx2e}
\usepackage{graphicx}
\usepackage{longtable}
\usepackage{float}
\usepackage{wrapfig}
\usepackage{soul}
\usepackage{textcomp}
\usepackage{marvosym}
\usepackage{wasysym}
\usepackage{latexsym}
\usepackage{amssymb}
\usepackage{hyperref}
\tolerance=1000
\usepackage[paperwidth=8.5in,paperheight=11in]{geometry}
\geometry{verbose,tmargin=0.5in,bmargin=1in,lmargin=1in,rmargin=1in}
\providecommand{\alert}[1]{\textbf{#1}}

\title{\vspace{-0.5in}Course Outcomes for STAT 5848 \& 6948}
%\author{G. Jay Kerns}
\date{\vspace{-0.5in}Summer 2012}
\hypersetup{
  pdfkeywords={},
  pdfsubject={},
  pdfcreator={Emacs Org-mode version 7.8.02}}

\begin{document}

\maketitle


\section*{Chapter 1}
\label{sec-1}

\begin{itemize}
\item Course policies, procedures, and expectations
\item Beginning concepts of time series, history
\item Get the computer up and running
\end{itemize}
\section*{Chapter 2: Basic concepts of mean, covariance, correlation, and stationarity}
\label{sec-2}

\begin{itemize}
\item Basic concepts of stochastic processes
\item Mean functions, autocovariance functions, and autocorrelation functions
\item Examples: white noise, random walk, moving average, random cosine wave
\item Introduce concept of stationarity
\end{itemize}
\section*{Chapter 3: Trend analysis, estimation and diagnostics}
\label{sec-3}

\begin{itemize}
\item Describing, modeling, and estimating deterministic trends
\item Constant mean: estimation and assessment of accuracy
\item Regression methods to estimate linear or quadratic trends
\item Methods for modeling cyclical or seasonal trends
\item Reliability and efficiency of all the above
\item Introduce residual analysis
\item Sample autocorrelation function
\end{itemize}
\section*{Chapter 4: Autoregressive moving average (ARMA) parametric models for stationary time series}
\label{sec-4}

\begin{itemize}
\item Introduce ARMA models
\item Special cases: MA(1) and MA(2)
\item Special cases: AR(1) and AR(2)
\item Stationarity and Invertibility issues
\item Properties of mixed ARMA models
\item Autocorrelation properties and the various representations
\end{itemize}
\section*{Chapter 5: Extend to ARIMA models for certain types of nonstationarity}
\label{sec-5}

\begin{itemize}
\item Use differencing to introduce stationary in nonstationary series.
\item Properties and behavior of ARIMA models
\item Other transformations such as logarithmic and percentage change
\item Power transformations, Box-Cox transformations to help with stationarity and/or normality
\end{itemize}
\section*{Chapter 6: Tentative specification of ARIMA models}
\label{sec-6}

\begin{itemize}
\item Specifying reasonable but simple models for observed times series
\item Tools for choosing the orders (p,d,q) for ARIMA(p,d,q) models
\item Sample autocorrelation function, sample partial autocorrelation function, sample extended autocorrelation function
\item Dickey-Fuller unit-root test to help distinguish (non)stationary series
\end{itemize}
\section*{Chapter 7: Efficient estimation of ARIMA model parameters}
\label{sec-7}

\begin{itemize}
\item Estimation of ARIMA model parameters
\item Criteria based on method of moments, least squares, and likelihood function
\item Properties of various estimators
\item Bootstrapping with ARIMA models
\end{itemize}
\section*{Chapter 8: Assessing ARIMA model fit}
\label{sec-8}

\begin{itemize}
\item Considerable expansion of residual analysis
\item Plots assess error terms for constant variance, normality, and independence
\item Properties of the sample autocorrelation of the residuals
\item Ljung-Box statistic portmanteau test as summary of residual autocorrelation
\item Overfitting and parameter redundancy
\end{itemize}
\section*{Chapter 9: Theory/methods of ARIMA forecasting}
\label{sec-9}

\begin{itemize}
\item Forecasting future values based on minimizing mean square forecasting error
\item Simple cases, extrapolate estimated trend
\item Autocorrelation, forecasts incorporate those
\item ARIMA forecasts: computation and properties
\item Prediction limits assess potential accuracy
\item Forecasts involving transformation of original series
\end{itemize}
\section*{Chapter 10: Extend all the above to seasonal ARIMA models}
\label{sec-10}

\begin{itemize}
\item Multiplicative seasonal ARIMA models
\item Identification, estimation, diagnostics
\end{itemize}
\section*{Chapter 11: Intervention analysis, outliers, spurious correlation, and prewhitening}
\label{sec-11}

\begin{itemize}
\item intervention models incorporate known external events which have a significant effect on the time series
\item Develop models to detect and incorporate outliers in time series
\item spurious correlation and its effects
\item prewhitening and stochastic regression
\end{itemize}
\section*{Chapter 12: Models for heteroscedasticity}
\label{sec-12}

\begin{itemize}
\item terms and issues associated with financial time series
\item Autoregressive conditional heteroscedasticity (ARCH) models
\item Special case: ARCH(1)
\item GARCH(p,q): generalized autoregressive conditional heteroscedasticity
\item identification, maximum likelihood estimation, prediction, and model diagnostics
\end{itemize}
\section*{Chapter 13: Spectral analysis}
\label{sec-13}

\begin{itemize}
\item Model as linear combinations of sines and cosines
\item Periodogram measures contribution of frequencies in spectral representation
\item Modeling with continuous range of frequencies
\item Spectral densities of ARMA models
\item Sampling properties of the sample spectral density
\end{itemize}
\section*{Chapter 14: Better methods for spectral density estimation}
\label{sec-14}

\begin{itemize}
\item Introduce smoothed sample spectral density
\item Bias, variance, leakage, bandwidth, and tapering
\item Procedure for forming confidence intervals
\end{itemize}
\section*{Chapter 15: Threshold models and nonlinear predictors}
\label{sec-15}

\begin{itemize}
\item Threshold model: nonlinear time series
\item How to test for nonlinearity and threshold nonlinearity
\item Parameter estimation using minimum AIC (MAIC) criterion and conditional least squares
\item Model diagnostics and extended portmanteau test
\item Forecasts and prediction intervals from threshold models
\end{itemize}



  

\end{document}