% Created 2012-07-18 Wed 15:35
\documentclass[12pt,english]{article}
\usepackage[utf8]{inputenc}
\usepackage[T1]{fontenc}
\usepackage{fixltx2e}
\usepackage{graphicx}
\usepackage{longtable}
\usepackage{float}
\usepackage{wrapfig}
\usepackage{soul}
\usepackage{textcomp}
\usepackage{marvosym}
\usepackage{wasysym}
\usepackage{latexsym}
\usepackage{amssymb}
\usepackage{hyperref}
\tolerance=1000
\usepackage[paperwidth=8.5in,paperheight=11in]{geometry}
\geometry{verbose,tmargin=0.5in,bmargin=1in,lmargin=1in,rmargin=1in}
\providecommand{\alert}[1]{\textbf{#1}}

\title{STAT 5848/6948 | Applied Regression \& Time Series}
%\author{G. Jay Kerns}
\date{\vspace{-0.5in}}
\hypersetup{
  pdfkeywords={},
  pdfsubject={},
  pdfcreator={Emacs Org-mode version 7.8.02}}

\begin{document}

\maketitle

\thispagestyle{empty}

The purpose of this assignment is to give you practice identifying ARIMA models.  I have generated data according to assorted ARIMA models, and your goal is to use what you have learned to identify the correct model.

\section*{The Data: \texttt{identifyARIMA.RData}}
\label{sec-1}


I will send the above file to your YSU email address. It is an \texttt{R} workspace which contains five (5) generated time series, namely, \texttt{x1}, \texttt{x2}, \ldots{}, \texttt{x5}.  In order to complete the assignment, you will need to download the workspace and load it into \texttt{R} in the following way:
\begin{enumerate}
\item Download and save \texttt{identifyARIMA.RData} somewhere on your computer.
\item Open \texttt{RStudio}, and go to \texttt{Workspace -> Load Workspace...}.
\item Navigate to wherever it was you saved \texttt{identifyARIMA.RData} and click \texttt{Open}.
\end{enumerate}
That's it.  If you type \texttt{x1} at the command prompt you should see a bunch of numbers.
\section*{The Assignment:}
\label{sec-2}


\noindent
For each time series \texttt{x1}, \texttt{x2}, \ldots{}, \texttt{x5}, your goal is to identify and specify the theoretical ARIMA model that generated it.  Here are the tools at your disposal:
\begin{itemize}
\item Plots of the original time series and/or differences
\item Lag plots of assorted orders
\item Sample ACF and Sample PACF
\item Extended ACF for mixed models
\end{itemize}
\textbf{For each time series} I am looking for three things:
\begin{enumerate}
\item The model in words, for example, AR(2).
\item The mathematical formula for your model.  For example, an AR(2) model with mean zero looks like this:
   \begin{equation}
   Y_{t} = \phi_{1}Y_{t - 1} + \phi_{2}Y_{t - 2} + e_{t}
   \end{equation}
   You can get this formula in your \texttt{Sweave} report with the following code:
\begin{verbatim}
    \begin{equation}
    Y_{t} = \phi_{1}Y_{t - 1} + \phi_{2}Y_{t - 2} + e_{t}
    \end{equation}
\end{verbatim}
\item Any supporting arguments, work, and/or plots you used to come to your conclusion.
\end{enumerate}
\section*{Assignment Deliverables:}
\label{sec-3}

Email your \texttt{Sweave}  \texttt{.Rnw} file to me at \texttt{gkerns@ysu.edu} by the due date, \textbf{Monday, July 23, 2012 @ 13:00}.  (Early submission is encouraged.)
\section*{Hints:}
\label{sec-4}

\begin{itemize}
\item Try figuring out \texttt{x5} first.
\item It is not necessary to transform any of the data with \texttt{sqrt}, \texttt{log}, etc.
\item To take the first difference $\nabla X_{t}$ do \texttt{diff(x)}.  To take the second difference  $\nabla^{2} X_{t}$ do \texttt{diff(diff(x))} or \texttt{diff(x, differences = 2)}, and so forth.
\item The $\LaTeX$ code for $\theta$, $\mu$, \emph{etc.} is \texttt{\textbackslash{}theta}, \texttt{\textbackslash{}mu}, \emph{etc.}  The code for $\nabla$ and  $\nabla^{2}$ is \texttt{\textbackslash{}nabla} and \texttt{\textbackslash{}nabla\textasciicircum{}\{2\}}.
\end{itemize}
\section*{Making your report look pretty:}
\label{sec-5}

The more comfortable you get working with \texttt{Sweave} files and $\LaTeX$, the more professionalism I am going to expect out of your reports.  For instance, for this assignment, everybody should have at bare minimum a title, author, and clearly delineated problems.

Everybody should have the following line somewhere in their report (above the \texttt{\textbackslash{}begin\{document\}} line, preferably).
\begin{verbatim}
 \title{Identifying ARIMA Models Assignment}
\end{verbatim}

The report should also say who wrote it, with a line somewhere that says something like this:
\begin{verbatim}
 \author{Sir Ronald Aylmer Fisher}
\end{verbatim}

You can divide your report up into sections, say, with different sections addressed to different problems.  For instance, you could do something like this:
\begin{verbatim}
 \section{The simulated time series \texttt{x1}}

 Here are a bunch of code chunks and plots about \texttt{x1}.


 \section{The simulated time series \texttt{x2}}

 Here are a bunch of code chunks and plots about \texttt{x2}.

 And so forth...
\end{verbatim}

Some of you are pretty good with $\LaTeX$ already and do not need additional guidance;  make your report as professional as you would anyway.

\end{document}